% !TEX root = ./rep.tex

\section*{Executive Summary}

The new DOE-NE Center of Excellence for Thermal-fluids applications in Nuclear Energy" inaugurated in
April 2018 considers and researches novel new solution strategies for historically challenging
flow issues that still plague the current  fleet of deployed Light Water Reactor (LWR) nuclear reactors as well as predicting various fluid flow and  fluid related issues with advanced reactor technologies. Our advanced thermal-fluids research and development approach synergistically combines three natural, though overlapping, length and time scales in a hierarchal multi-scale approach to avoid the temptation and pitfalls of attempting to develop a single solve all algorithm for physical  fluid flow problems that will span $10^9$ in spatial and temporal scales.

The demonstrate the multi-scale philosophy of the center we focus on Flouride Cooled High
Temperature Reactors (FHRs), and in particular on the Berkley's PB-FHR design Mark-I design. The
Fluoride salt cooled High temperature Reactor (FHR) is a class of advanced nuclear reactors that combine the
robust coated particle fuel form from high temperature gas cooled reactors, direct reactor auxiliary cooling
system (DRACS) passive decay removal of liquid metal fast reactors, and the transparent, high volumetric heat capacitance liquid Fluoride salt working fluids - Flibe - from molten salt reactors. This combination of fuel and coolant enables FHRs to operate in a high-temperature low-pressure design space that has beneficial safety and economic implications. The PB-FHR reactor relies on a pebble bed approach and pebble bed reactors are in a sense the poster child for multiscale analysis.

The lower length-scale simulator for pebble reactor cores comprises three physics: neutronics (OpenMC),
thermal-fluids (Nek5000/NekRS) and fuel performance (BISON). As part of center of excellence ongoing research efforts we have developed Cardinal, a new tool platform for lower length-scale simulation. Cardinal tightly couples all three physics and leverages advances in MOOSE such as the MultiApp system and the concept of MOOSE-wrapped Apps. The present reports provides an update on the development of Cardinal with an extension of Cardinal to GPUs. We also perform a first-of-a-kind demonstration simulation on Summit representing a $10x$ capability increase in terms of pebble count for Cardinal.
