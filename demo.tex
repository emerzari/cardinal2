% !TEX root = ./rep.tex
\section{Demonstration simulation}
\label{s:demo}

To demonstrate the new capability of Cardinal we demonstrate here the use of Cardinal on Summit to perform coupled multi-physics simulations of pebble beds.

\subsection{Numerical Setup}

The case presented here comprises 1568 pebbles. The pebble configuration has been obtained using a discrete element method (DEM) code. A major overhaul on the mesh generation  for the fluid domain has been necessary to automatize as much as possible the process and reduce the number of elements per pebble. The new meshing tool is based on a Voronoi cell strategy. It has allowed to produce high quality hexahedral meshes, while reducing the mesh count to roughly 300-400 elements per pebble.

Examples for these meshes are provided in Figure~\ref{f:ndemo1} for the 1568 pebble configuration and in Figure~\ref{f:ndemo2} for a pebble configuration with over 3,000 pebbles. Both meshes have a similar density.

We note that 1568 pebbles is a significant size for a coupled calculation and representative for instance of the SANA experiments.

\begin{figure}[!h]
\centering
\includegraphics[clip=true,width=0.9\textwidth]{Figures/ndemo_r1}
\caption{NekRS Mesh for 1568 pebble configuration}
\label{f:ndemo1}
\end{figure}

\begin{figure}[!h]
\centering
\includegraphics[clip=true,width=0.9\textwidth]{Figures/ndemo_r2}
\caption{NekRS Mesh for 3260 pebble configuration}
\label{f:ndemo2}
\end{figure}

For this demonstrations, the sizes and composition of the TRISO particles were based on TRISO manufactured
at INL, following the practice established in the previous report \cite{cardinal}. Though these particles were developed for the Advanced Gas Reactor (AGR) fuel, particles with the same specifications are used for FHR test reactors and computations benchmarks. The sizes and compositions of the pebbles were taken from the Mark-1 FHR reactor constructed at UC Berkeley.

For BISON, and the demonstration problem under consideration we consider only the conduction equation and as such it is a relatively straightforward setup. Properties are constant and adapted from available correlations. The mesh for a single sphere is generated and replicated at run time. The same mesh is used for the mesh tallies in OpenMC.

\subsection{Results}

The model described in the previous section has been run on 12 and 20 nodes of Summit, with 6 MPI ranks on each node, corresponding to the 6 GPU on each node. The OpenMC and BISON models are designed to run on CPU while the NekRS model runs on the GPU.

Stand-alone NekRS simulations have been run first up to 25 convective time units to develop turbulence - Figure~\ref{f:ndemo3} using a Large Eddy Simulation (LES) approach.

 A restart file has then been generated and used to restart a transient simulation in Cardinal representing an heat-up of the pebbles. The time step has been fixed to $5*10^-4 s$ in both BISON and NekRS. The temperature at time zero has been set to 300 C everywhere. Figure~\ref{f:ndemo4} presents the temperature at the surface of the pebbles in BISON at three points in time. The simulations took 2.5s per coupled time step on 20 nodes, requiring transfer between physics at each time step. However this could be greatly optimized by relaxing the requirement, as data transfer from GPU to CPU should be minimized as much as possible.

\begin{figure}[!h]
\centering
\includegraphics[clip=true,width=0.9\textwidth]{Figures/ndemo_r3}
\caption{NekRS results for the velocity field.}
\label{f:ndemo3}
\end{figure}


\begin{figure}[!h]
\centering
\includegraphics[clip=true,width=0.9\textwidth]{Figures/ndemo_r4}
\caption{Temperature result for the pebble surface temperature at three points in time.}
\label{f:ndemo4}
\end{figure}

\subsection{Projection to full core}

20 nodes of Summit represents less than 1\% of the computing power available on Summit. We estimate that 80\% of the machine will be sufficient to perform full core calculation in FHRs corresponding to 300,000 pebbles.
